% !TEX root = ../main.tex

\hitszsetup{
  %******************************
  % 注意:
  %   1. 配置里面不要出现空行
  %   2. 不需要的配置信息可以删除
  %******************************
  %
  %=====
  % 秘级
  %=====
  statesecrets={公开},
  natclassifiedindex={TM301.2},
  intclassifiedindex={62-5},
  %
  %=========
  % 中文信息
  %=========
  ctitleone={基于神经网络的机器人},%本科生封面使用
  ctitletwo={智能抓取研究},%本科生封面使用
  ctitlecover={基于神经网络的机器人智能抓取研究},%放在封面中使用,自由断行
  ctitle={基于神经网络的机器人智能抓取研究},%放在原创性声明中使用
  csubtitle={一条副标题}, %一般情况没有,可以注释掉
  cxueke={工学},
  csubject={机械设计制造及其自动化},
  % csubject={机械工程},
  caffil={机电工程与自动化学院},
  cauthor={杨敬轩},
  csupervisor={某某某 教授},
  cassosupervisor={某某某 教授}, % 副指导老师
  % ccosupervisor={某某某 教授}, % 联合指导老师
  % 日期自动使用当前时间,若需指定按如下方式修改:
  %cdate={超新星纪元},
  cstudentid={SZ160310217},
  cstudenttype={同等学力人员}, %非全日制教育申请学位者
  %(同等学力人员)、(工程硕士)、(工商管理硕士)、
  %(高级管理人员工商管理硕士)、(公共管理硕士)、(中职教师)、(高校教师)等
  %
  %
  %=========
  % 英文信息
  %=========
  etitle={Research on robot intelligent grasping based on Neural Network},
  esubtitle={This is the sub title},
  exueke={Engineering},
  esubject={Mechanical Engineering},
  eaffil={Harbin Institute of Technology, Shenzhen},
  eauthor={Jingxuan Yang},
  esupervisor={Prof. XXX},
  % eassosupervisor={XXX},
  % 日期自动生成,若需指定按如下方式修改:
  edate={June, 2020},
  estudenttype={Master of Engineering},
  %
  % 关键词用“英文逗号”分割
  ckeywords={关键词1, 关键词2, 关键词3, 关键词4, 关键词5},
  ekeywords={keyword 1, keyword 2, keyword 3, keyword 4, keyword 5},
}

% 中文摘要
\begin{cabstract}

  摘要是论文内容的高度概括,应具有独立性和自含性,即不阅读论文的全文,就能通过摘要了解整个论文的必要信息。摘要应包括本论文的目的、理论与实际意义、主要研究内容、研究方法等,其中重点突出研究成果和结果。

  摘要中不宜使用公式、化学结构式、图表和非公知公用的符号和术语,不标注引用文献编号。摘要的内容要完整、客观、准确,应做到不遗漏、不拔高、不添加,避免将摘要写成目录式的内容介绍。摘要在叙述研究内容、研究方法和主要结论时,除作者的价值和经验判断可以使用第一人称外,一般使用第三人称,采用“分析了XXX原因”、“认为XXX”、“对XXX进行了探讨”等记述方法进行描述。避免主观性的评价意见,避免对背景、目的、意义、概念和一般性(常识性)理论叙述过多。

  关键词在正文之后隔一行顶格书写。各关键词之间用分号,换行缩进对齐,最后一个关键词后不加标点。

  {\color{red}(关键词是供检索用的主题词条。关键词应集中体现论文特色,反映研究成果的内涵,具有语义性,在论文中有明确的出处,并应尽量采用《汉语主题词表》或各专业主题词表提供的规范词,应列取3至6个关键词,按词条的外延层次从大到小排列)}

\end{cabstract}

% 英文摘要
\begin{eabstract}

  Abstract is a highly generalization of the content of the paper, which should be independent and self-contained, that is, without reading the full text of the paper, we can understand the necessary information of the whole paper through the abstract. It should include the purpose, theoretical and practical significance, main research contents, research methods, etc. of this paper, especially the research results and results.

  It is not suitable to use formula, chemical structure formula, chart and non-public symbols and terms in the abstract, without reference number. The content of the abstract should be complete, objective and accurate, and should not be omitted, promoted or added, so as to avoid the introduction of the abstract as a table of contents. In describing the research content, research methods and main conclusions, except the author’s value and experience judgment, the third person is generally used. Avoid subjective evaluation opinions and excessive narration of background, purpose, significance, concept and general (common sense) theory.

  Interlace the body of the abstract. Key words are written at the top of the text. Use semicolons between keywords, line feed indents to align, and do not punctuate the last keyword.

\end{eabstract}
