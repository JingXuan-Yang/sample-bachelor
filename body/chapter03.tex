% !TEX root = ../main.tex

\chapter{内容XXX}

\section{引言}

每章的引言起到承接上一章引启下一章的作用。

\ldots\ldots

\section{对物理量符号进行注释的情况}

为使得对公式中物理量符号注释的转行与破折号“———”后第一个字对齐,此处最好采用表格环境。此表格无任何线条,左对齐,且在破折号处对齐,一共有“式中”二字、物理量符号和注释三列,表格的总宽度可选为文本宽度,因此应该采用\verb|tabularx|环境。由\verb|tabularx|环境生成的对公式中物理量符号进行注释的公式如式(\ref{eq:1})所示。
\begin{equation}\label{eq:1}
\ddot{\bm{\rho}}-\frac{\mu}{R_{t}^{3}}\left(3\bm{R_{t}}\frac{\bm{R_{t}\rho}}{R_{t}^{2}}-\bm{\rho}\right)=\bm{a}
\end{equation}
\begin{tabularx}{\textwidth}{@{}l@{\quad}r@{———}X@{}}
式中& $\bm{\rho}$ &追踪飞行器与目标飞行器之间的相对位置矢量;\\
&  $\bm{\ddot{\rho}}$&追踪飞行器与目标飞行器之间的相对加速度;\\
&  $\bm{a}$   &推力所产生的加速度;\\
&  $\bm{R_t}$ & 目标飞行器在惯性坐标系中的位置矢量;\\
&  $\omega_{t}$ & 目标飞行器的轨道角速度;\\
\end{tabularx}\vspace{3.15bp}
由此方法生成的注释内容应紧邻待注释公式并置于其下方,因此不能将代码放入\verb|table|浮动环境中。但此方法不能实现自动转页接排,可能会在当前页剩余空间不够时,全部移动到下一页而导致当前页出现很大空白。因此在需要转页处理时,还请您手动将需要转页的代码放入一个新的\verb|tabularx|环境中,将原来的一个\verb|tabularx|环境拆分为两个\verb|tabularx|环境。

{\color{red}(矩阵、矢量用“粗、斜体”,如矢量$\bm{R}$;单变量用“斜体”(不加粗),如$x,y$;上下标:有变量含义的用斜体,如$x_i$;数字、单词首字母、单位等无变量含义的用正体,如$x_1$,矩阵转置$\bm{A}^{\text{T}}$(T 为转置Transpose 的首字母))}

\section{子公式}

子公式编号示例:如果需要对公式的子公式进行编号,则使用\lstinline{subeqnarray}环境:
\begin{subeqnarray}
  \label{eqw}
  \slabel{eq0}
  x & = & a \times b \\
  \slabel{eq1}
  & = & z + t\\
  \slabel{eq2}
  & = & z + t
\end{subeqnarray}

\equref{eqw}中,\lstinline{label}为整个公式的标签,\lstinline{slabel}为子公式的标签。

\section{本章小结}

总结本章的叙述内容。

\lipsum[1]
